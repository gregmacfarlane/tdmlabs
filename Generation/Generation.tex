There are two weeks to this unit. In the first week, you will build linear
regression and cross-classification models to predict household trips. In the
second week, you will use the models you develop in the Wasatch Front Travel
Demand Model.


\section{Model Generation}
\subsection{Linear Regression Models}
In the first part of this assigment, you will build a linear model to predict
home-based work trips, and a cross-classification model to predict three
different types of trips. You will then compare these two models to see which
variables significantly predict different types of trips. We will use the 
\texttt{TripGen.Rdata} file which was created from the NHTS.

\begin{knitrout}
\definecolor{shadecolor}{rgb}{0.969, 0.969, 0.969}\color{fgcolor}\begin{kframe}
\begin{alltt}
\hlkwd{library}\hlstd{(gdata)}
\hlkwd{load}\hlstd{(}\hlstr{"./TripGen.Rdata"}\hlstd{)}
\end{alltt}
\end{kframe}
\end{knitrout}


Build a linear regression model to predict the number of household work trips.
Include a model with just the number of workers (\verb#WRKCOUNT#), and one with just
the number of automobiles (\verb#HHVEHCNT#), and one with severable variables
that you find interesting and significant. Ensure to treat categorical variables
appropriately (\verb#factor()#), and eliminate missing information from your
model variables (\verb#unknownToNA#). Provide a residual plot of your final
model.

Build a linear regression model that incorporates the household income
alone, \verb#HHFAMINC#. You should code this variable so that it represents
actual dollar values, in thousands of dollars. A mechanism for doing this might
be 
\begin{knitrout}
\definecolor{shadecolor}{rgb}{0.969, 0.969, 0.969}\color{fgcolor}\begin{kframe}
\begin{alltt}
\hlstd{missing.values} \hlkwb{<-} \hlkwd{c}\hlstd{(}\hlopt{-}\hlnum{9}\hlstd{,} \hlopt{-}\hlnum{8}\hlstd{,} \hlopt{-}\hlnum{7}\hlstd{)}
\hlstd{TripGen}\hlopt{$}\hlstd{HHFAMINC} \hlkwb{<-} \hlkwd{unknownToNA}\hlstd{(TripGen}\hlopt{$}\hlstd{HHFAMINC, missing.values)}
\hlstd{TripGen}\hlopt{$}\hlstd{HHINCOME} \hlkwb{<-} \hlkwd{ifelse}\hlstd{(TripGen}\hlopt{$}\hlstd{HHFAMINC} \hlopt{==} \hlnum{18}\hlstd{,} \hlnum{100}\hlstd{,} \hlkwd{ifelse}\hlstd{(TripGen}\hlopt{$}\hlstd{HHFAMINC} \hlopt{==}
    \hlnum{17}\hlstd{,} \hlnum{90}\hlstd{, (TripGen}\hlopt{$}\hlstd{HHFAMINC} \hlopt{*} \hlnum{5} \hlopt{-} \hlnum{2.5}\hlstd{)))}
\end{alltt}
\end{kframe}
\end{knitrout}


After this model, include income in a model controlling for other variables such
as the number of workers in a household. What happens to the sign of the income
paramater estimate? How do you explain this?



\subsection{Cross-classification Models}\label{sec:crossclass}
In this portion of the assignment, you will build cross-classification models to
predict household trips. A cross classification model simply
assigns an expected value to a household, conditioned on one or more variables. 
Because it may not be appropriate to estimate mean values for classes that are 
unlikely to be represented, you should cap the number of vehicles or adults 
that you estimate categories for.
\begin{knitrout}
\definecolor{shadecolor}{rgb}{0.969, 0.969, 0.969}\color{fgcolor}\begin{kframe}
\begin{alltt}
\hlstd{TripGen}\hlopt{$}\hlstd{adultclass} \hlkwb{<-} \hlkwd{ifelse}\hlstd{(TripGen}\hlopt{$}\hlstd{NUMADLT} \hlopt{>} \hlnum{6}\hlstd{,} \hlnum{6}\hlstd{, TripGen}\hlopt{$}\hlstd{NUMADLT)}
\hlstd{TripGen}\hlopt{$}\hlstd{vehclass} \hlkwb{<-} \hlkwd{ifelse}\hlstd{(TripGen}\hlopt{$}\hlstd{HHVEHCNT} \hlopt{>} \hlnum{3}\hlstd{,} \hlnum{3}\hlstd{, TripGen}\hlopt{$}\hlstd{HHVEHCNT)}
\end{alltt}
\end{kframe}
\end{knitrout}


Calculate the average number of HBO trips per household, conditioned on
the household size and number of household vehicles. Present your results in a
table. A good way to do this is with the \verb#aggregate()# function:
\begin{knitrout}
\definecolor{shadecolor}{rgb}{0.969, 0.969, 0.969}\color{fgcolor}\begin{kframe}
\begin{alltt}
\hlstd{hboaverage} \hlkwb{<-} \hlkwd{aggregate}\hlstd{(TripGen}\hlopt{$}\hlstd{HBOTrips,} \hlkwc{by} \hlstd{=} \hlkwd{list}\hlstd{(TripGen}\hlopt{$}\hlstd{adultclass, TripGen}\hlopt{$}\hlstd{vehclass),}
    \hlkwc{FUN} \hlstd{= mean)}
\end{alltt}
\end{kframe}
\end{knitrout}

Provide similar tables for HBW and NHB trips. Are these numbers
reasonable? Submit a memorandum outlining your work. Attach a script 
containing the R  commands you used in your analysis.

\clearpage
\section{Model Implementation}
Replace the trip generation rates in the Wasatch Front Travel Demand Model with
the rates you calculated in Section \ref{sec:crossclass}. The script necessary
is \verb#1TripGen.s#; you should copy this file and save the original as
\verb#1TripGen(ORIGINAL).s#. The rates are discussed on page 36 of the model
documentation. You should notice that HBO and NHB trips are divided into
further subcategories, 
\[HBO = \text{hbotp} + \text{hbscp} + \text{hbshp} + \text{hbpbp}\]
Because we did not calculate each individual HBO trip sub-type, you should develop a
weighting system to distribute your estimates across the sub-types. Any method
is appropriate, but clearly document what you do.

Run a 2009 scenario with the altered trip generation rates, and compare the
principal log files with the base model you computed in Lab \ref{lab:run_model}.
Discuss the linearity of changes to the trip generation rates; that is, does a
1\% decrease in the average generation rate correspond to a 1\% decrease in
region trips? Why or why not?

Comment on the adequacy of the National Household Travel Survey as a replacement
for locally collected surveys. Submit your analysis in a technical memorandum.
